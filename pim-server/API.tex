\documentclass{article}

\usepackage{xeCJK}
\setCJKmainfont{SimSun}

\title{Personal Information Manager System API Document}
\author{李约瀚 \\ 14130140331 \\ qinka@live.com \\ qinka@qinka.pw}

\usepackage{listings}
\usepackage{hyperref}

% For Haskell
\lstnewenvironment{haskell}[1][]
{\lstset{ frame=tblr
        , breaklines           
        , basicstyle=\small\ttfamily  
        , flexiblecolumns=false  
        , basewidth={0.5em,0.45em}   
        , literate={+}{{$+$}}1 % plus  
            {/}{{$/$}}1                  
            {*}{{$*$}}1                  
            {=}{{$=$}}1                  
            {>}{{$>$}}1                  
            {<}{{$<$}}1                  
            {\\}{{$\lambda$}}1           
            {\\\\}{{\char`\\\char`\\}}1  
            {->}{{$\rightarrow$}}2       
            {>=}{{$\geq$}}2              
            {<-}{{$\leftarrow$}}2        
            {<=}{{$\leq$}}2              
            {=>}{{$\Rightarrow$}}2       
            {\ .}{{$\circ$}}2            
            {\ .\ }{{$\circ$}}2          
            {>>}{{>>}}2                  
            {>>=}{{>>=}}2                
            {|}{{$\mid$}}1               
            {/=}{{$\neq$}}1              
        , numbers=left%
        , #1%
    }%
    \csname lst@SetFirstLabel\endcsname}
{\csname lst@SaveFirstLabel\endcsname}
% For Java
\lstnewenvironment{java}[1][]
{\lstset{frame=tblr
        , breaklines           
        , basicstyle=\small\ttfamily  
        , flexiblecolumns=false  
        , basewidth={0.5em,0.45em} 
        , literate={+}{{$+$}}1 % plus  
                   {/}{{$/$}}1                  
                   {*}{{$*$}}1                  
                   {=}{{$=$}}1                  
                   {>}{{$>$}}1                    
                   {<}{{$<$}}1       
                   {>=}{{$\geq$}}2        
                   {<=}{{$\leq$}}2             
        , numbers=left%
        , #1
        }
    \csname lst@SetFirstLabel\endcsname}
{\csname lst@SaveFirstLabel\endcsname}
% For JSON
\lstnewenvironment{json}[1][]
{\lstset{ frame=tblr
        , breaklines           
        , basicstyle=\small\ttfamily  
        , flexiblecolumns=false  
        , basewidth={0.5em,0.45em}             
        , numbers=left%
        , #1%
        }
    \csname lst@SetFirstLabel\endcsname}
{\csname lst@SaveFirstLabel\endcsname}


\begin{document}
    \maketitle
    \newpage
    \tableofcontents
    \newpage
    
    % % Main
    \section{Application Text Data Interface}
    This section includes the application text data interfaces' definition.
    At the same time, this section will include the representation application datas' typeclass/interface.
    
    For the common texture data interface, JSON will be used when datas are needed to be represent to text.
    
    \subsection{[Typeclass/Interface] Dateable}
    If the tuple $(D,O)$ is dateable, the operator $O$, which also should be a function, 
    map on the collection $D$, and that should get the dates of $D$.
    
    For Haskell:
    \begin{haskell}
 class Dateable a where
   getDate :: a -> Date
    \end{haskell}
    and, for Java:
    \begin{java}
  interface IDateable
  {
      public abstract Date getDate();
  }
    \end{java}
    
    \subsection{[Typecalss/Interface] Itemable}
    If the tuple $(D,P_{set},P_{get},C_{set},C_{get})$ is itemable, the operators $P$, a function, map on the collection $D$,
    and that should get the priorities of $D$ or set the priorities of $D$, as well as, the operators $C$,
    a function too, map on the collection $D$, and that should get the contexts of $D$ or set the contexts of $D$.
    
    For Haskell:
    \begin{haskell}
 class Itemable a where
   getPriority :: a -> Priority
   setPriority :: a -> Priority -> a
   getContext  :: a -> Context
   setContext  :: a -> Context  -> a
    \end{haskell}
    and, for Java:
    \begin{java}
  interface IItemable
  {
      public abstract Priority getPriority();
      public abstract void     setPriority(Priority);
      public abstract Context  getContext();
      public abstract void     setContext(Context);
  }
    \end{java}
    
    \subsection{[Data] Todo}
    The \lstinline|Todo| or \lstinline|PIMTodo| is a data structure, or say class, which hold the todo within context,
    and this should be the instance of both Dateable and Itemable.
    
    For the common text interface, the followings are necessary field: 
    \begin{json}
 {
     "type":"todo",
     "date":"{DATE}",
     "context:"{STRING}"
 }
    \end{json} 
    
    \subsection{[Data] Note}
    The \lstinline|Note| or \lstinline|PIMNote| is a data structure, or say class, which hold the note within context,
    and this should be the instance of Itemable.
    
    For the common text interface, the followings are necessary field: 
    \begin{json}
 {
     "type":"note",
     "context":"{STRING}",
 }
    \end{json}
    
    \subsection{[Data] Appointment}
    The \lstinline|Appointment| or \lstinline|PIMAppointment| is a data structure, or say class,
    which hold the appointment informations within the context, and this should be instance of both Datebale and Itemable.
    
    For the common text interface, the followings are necessary field: 
    \begin{json}
 {
     "type":"appointment",
     "date":"{DATE}",
     "context":"{STRING}"
 }
    \end{json}
    
    \subsection{[Data] Contact}
    The \lstinline|Contact| or \lstinline|PIMContact| is a data structure, or say class, which hold the first name, last name,
    and email, and this should be the instance of Itemable.
    
    For the common text interface, the followings are necessary field: 
    \begin{json}
 {
     "type":"contact",
     "first-name":"{STRING}",
     "last-name":"{STRING}",
     "email":"{STRING}"
 }
    \end{json}
    
    
    \subsection{[Enum] Priority}
    The priority should have five level: \verb|right now|, \verb|urgent|, \verb|normal|, \verb|soon|, and \verb|never|.
    
    \section{Application Connection Interface}
    This section is about the standard actions for the client, which try to manage the information itself or the data itself.
    
    \subsection{Manager}
    The \lstinline|Manager| or \lstinline|PIMManager| should a data structure, or say class, which should have the following functions.
    
    If a tuple $(list,create,save,load,D)$(or say a subtuple), whose elements are the operators and a collection of items($D$), can meet the conditions:
    \begin{itemize}
        \item For $list$, the operator $list$ can show or display the list of the elements(items) in the collection($D$).
        \item For $create$, the operator $create$ can create or add an new element(item) to the collection($D$), and get a new collection.
        \item For $save$, the operator $save$ can use \verb|remote collection| to save the datas to wherever 
        \verb|remote collection| store them.
        \item For $load$, the operator $load$ can use \verb|remote collection| to load the datas from wherever
        \verb|remote collection| store them.
    \end{itemize}
    We can just call that tuple a $manager$. 
    
    If a \textit{super-manage} $(list,create,save,load,quit,D)$ has the operator $quit$ which can quit the current state machine,
    we can say this \textit{super-manage} is regarded as \textbf{\textit{Standard Manage Form}}. 
    
\end{document}