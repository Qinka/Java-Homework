\documentclass{article}

\usepackage{xeCJK}
\setCJKmainfont{SimSun}

\usepackage{listings}
\lstset{breaklines,numbers=left}

\usepackage{hyperref}

\usepackage{uml}

\title{Java Homework Report III}
\author{李约瀚 \\ 14130140331 \\ Qinka \\ qinka@live.com \\ me@qinka.pro}


\begin{document}

\maketitle
\newpage

{\Huge \textbf{Copyleft (C) 2017}}
\par
{\Huge \textbf{GPL 3}}
\newpage

\tableofcontents
\newpage

\section{Design}
\label{sec:design}

The Collection of the PIM, named \lstinline|PIMCollection|, is designed as a container, as well as a collection,
where the items will be stored.

The basic \lstinline|PIMCollection| should be looked like figure \ref{fig:uml:basic:pimc}.

\begin{figure}[h!]
  \centering
  \umlDiagram[box=,sizeX=7cm,sizeY=4cm]
  {
    \umlClass[reference=pimca,pos={1,1}]{PIMCollection}{}
    {%
      \umlMethod[visibility=+,returntype=Collection]{getNotes}{}
      \umlMethod[visibility=+,returntype=Collection]{getTodos}{}
      \umlMethod[visibility=+,returntype=Collection]{getAppointments}{}
      \umlMethod[visibility=+,returntype=Collection]{getContacts}{}
      \umlMethod[visibility=+,returntype=Collection]{getItemsForDate}{Date d}
    }
  }
  \caption{Basic PIMCollection}
  \label{fig:uml:basic:pimc}
\end{figure}

According to \lstinline|RemotePIMCollection| in final homework, the ``Collection'' is changed,
like figure \ref{fig:uml:pimc}.

\begin{figure}[h!]
  \centering
  \umlDiagram[box=,sizeX=9cm,sizeY=7cm]
  {
    \umlClass[reference=pimc,pos={1,1}]{PIMCollection}{}
    {%
      \umlMethod[visibility=+,returntype=Collection]{getNotes}{}
      \umlMethod[visibility=+,returntype=Collection]{getNotes}{String owner}
      \umlMethod[visibility=+,returntype=Collection]{getTodos}{}
      \umlMethod[visibility=+,returntype=Collection]{getTodos}{String owner}
      \umlMethod[visibility=+,returntype=Collection]{getAppointments}{}
      \umlMethod[visibility=+,returntype=Collection]{getAppointments}{String owner}
      \umlMethod[visibility=+,returntype=Collection]{getContacts}{}
      \umlMethod[visibility=+,returntype=Collection]{getContacts}{String owner}
      \umlMethod[visibility=+,returntype=Collection]{getItemsForDate}{Date d}
      \umlMethod[visibility=+,returntype=Collection]{getItemsForDate}{Date d,String owner}
      \umlMethod[visibility=+,returntype=Collection]{getAll}{}
      \umlMethod[visibility=+,returntype=Collection]{getAll}{String owner}
      \umlMethod[visibility=+,returntype=boolean]{add}{PIMEntity item}
    }
  }
  \caption{Class PIMCollection}
  \label{fig:uml:pimc}
\end{figure}


That class should be an inferface \lstinline|PIMBaseCollection|, just like figure \ref{fig:uml:ipimc}.

\begin{figure}[h!]
  \centering
  \umlDiagram[box=,sizeX=8.54cm,sizeY=12cm]
  {
    \umlClass[reference=collection,posDelta={5,28},stereotype=interface]{Collection<PIMEntity>}{}{...}
    \umlClass[reference=ipimc,posDelta={1,7},stereotype=interface]{PIMBaseCollection}{}
    {%
      \umlMethod[visibility=+,returntype=Collection]{getNotes}{}
      \umlMethod[visibility=+,returntype=Collection]{getNotes}{String owner}
      \umlMethod[visibility=+,returntype=Collection]{getTodos}{}
      \umlMethod[visibility=+,returntype=Collection]{getTodos}{String owner}
      \umlMethod[visibility=+,returntype=Collection]{getAppointments}{}
      \umlMethod[visibility=+,returntype=Collection]{getAppointments}{String owner}
      \umlMethod[visibility=+,returntype=Collection]{getContacts}{}
      \umlMethod[visibility=+,returntype=Collection]{getContacts}{String owner}
      \umlMethod[visibility=+,returntype=Collection]{getItemsForDate}{Date d}
      \umlMethod[visibility=+,returntype=Collection]{getItemsForDate}{Date d,String owner}
      \umlMethod[visibility=+,returntype=Collection]{getAll}{}
      \umlMethod[visibility=+,returntype=Collection]{getAll}{String owner}
      \umlMethod[visibility=+,returntype=boolean]{add}{PIMEntity item}
    }
    \umlClass[reference=pimc,posDelta={7,1}]{PIMCollection}{}{...}
    \umlInstance{pimc}{ipimc}
    \umlSubclass{ipimc}{collection}
  }
  \caption{Inferface of PIMCollection}
  \label{fig:uml:ipimc}
\end{figure}

When the ``Collection'' need to store datas to somewhere or fetch the datas from somewhere,
there is a interface, named \lstinline|PIMIOCollection|, needed,
and the UML figure is figure \ref{fig:uml:iointerface}.

\begin{figure}[h!]
    \centering
    \umlDiagram[box=,sizeX=7cm,sizeY=4cm]
    {%
        \umlClass[reference=collection,posDelta={4,2.6},stereotype=interface]{PIMIOCollection}{}
        {%
            \umlMethod[visibility=+,returntype=void]{push}{}
            \umlMethod[visibility=+,returntype=void]{pull}{}
        }
    }
    \caption{The Inferface of IO}
    \label{fig:uml:iointerface}
\end{figure}

In the interface \lstinline|PIMBaseCollection| the functions,
there, are used to get the items or add the item to collection.
When getting from the somewhere, the items can be filtered by ownership.
When storing the datas to disk, backend, or database, the interface \lstinline|PIMIOCollection|
define two method: \lstinline|push| and \lstinline|pull|. The former is updating datas to somewhere,
and the latter is fetching datas from somewhere. The build-in command \verb|save| and \verb|load|,
will call such two methods to update and fetch.

\section{Sources}
\label{sec:source}

The \lstinline|PIMBaseCollection| is an instance of \lstinline|Collection<PIMEntity|,
and at the same time, the basic \lstinline|PIMCollection| is an instance of \lstinline|ArratList<PIMEntity|.
The other collection instances is the subclass of the \lstinline|PIMCollection|.

The codes are in the folder.

\end{document}