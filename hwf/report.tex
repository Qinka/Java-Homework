\documentclass{article}

% xeCJK
\usepackage{xeCJK}
\setCJKmainfont{SimSun}

%listings
\usepackage{listings}
\lstset{breaklines,numbers=left}

% hypper link
\usepackage{hyperref}

\title{Java Homework Report for Final Project}
\author{李约瀚 \\ 14130140331 \\ Qinka \\ qinka@live.com \\ me@qinka.pro}

\begin{document}
\maketitle
\newpage

{\Huge \textbf{Copyleft (C) 2017}}
\par
{\Huge \textbf{GPL 3}}
\newpage

\tableofcontents
\newpage

\section{Design}
\label{sec:design}

For the Person Information Manager, there are three part: entity, collection, and GUI.

For the first part, entity, it is about the items of the personal informations. The informations are divided into four kinds: appointment, contact, note, and todo.
It was implemented in the homework 2. All there kinds of informations will be stored in file with JSON.

For the second part, collection, it is about how the items are stored, and interact with the remote storage.
There are three kinds of the ``remote'' storage: local file, database, web application backend.
It was implemented in the homework 2,3, and final one. Some of them were done with the after-class assignments.

For the third part, GUI, it is about the Graphical User Interface. There are the window for connecting with remote storage, the window for personal informations
in each month, and other windows for creating, showing details or something else.
It was implemented in homework of the final one.

All the items are derived from a abstract class, named \lstinline|PIMEntity|. Then the collection of the \lstinline|PIMEntity| can be defined.
The collection needs two functions: storing and interacting with remote. The collection should be about to support the normal operations of \lstinline|Collection|; meanwhile, it should be able to pull or push the datas from or to remote.

\subsection{How does it work?}
\label{sec:design:hiw}

When design this application, the first question, which should be asked, is how does it work, or say how does it work like.
Before the main window launching, the connection of remote should be established. So the first window is the ``Connection Window'',
where the connection type is selected, where the owner of the items is typed in, and where the params are filled.

Next one is the main window. In the main window, there should be panels, where the date and the items are displayed, head title, buttons of jumping,
the list of non-date items, and the menu.

For the items, there should be the dialog to display or create them.

\section{Implementing}
\label{sec:implement}

Because of the deadline of this homework, the date of test, and there are too many work needed to be done. For such a application's codes, you have to
find out them by yourself, sorry.

The full project and another homework can be found on \href{https://github.com/Qinka/Java-Homework}{GitHub}.

\section{Documents}
\label{sec:doc}

Another documents about this application can be found at the at \href{}{online} or just offline in the doc folder.

\end{document}

