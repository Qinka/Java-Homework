\documentclass{article}

\usepackage{xeCJK}
\setCJKmainfont{SimSun}

\usepackage{listings}
\lstset{breaklines,numbers=left}

\title{Java Homework Report II}
\author{李约瀚 \\ 14130140331 \\ Qinka \\ qinka@live.com \\ me@qinka.pro \\ qinka@qinka.pw (remove soon)}


\begin{document}

\maketitle
\newpage

{\Huge \textbf{Copyleft (C) 2017}}
\par
{\Huge \textbf{GPL 3}}
\newpage

\tableofcontents
\newpage

\section{Persional Information System}
\label{sec:pim}

\subsection{Task}
\label{sec:pim:task}

This assignment involves the creation of simple Personal Information Management system that can deal with 4 kinds of items:
todo items, notes, appointments and contacts. Each of these kinds of items is described in more detail below.
The assignment requires that you create a class for each item type, and that each class extends an abstract base class provided for you.
In addition to creating the four classes, you need to create a manager class that supports some simple text-based commands
for creating and managing items.

\paragraph{PIMTodo}

Todo items must be PIMEntites defined in a class named PIMTodo. Each todo item must have a priority (a string),
a date and a string that contains the actual text of the todo item.

\paragraph{PIMNote}

Note items must be PIMEntites defined in a class named PIMNote. Each note item must have a priority (a string),
and a string that contains the actual text of the note.

\paragraph{PIMAppointment}

Appointment items must be PIMEntites defined in a class named PIMAppointment. Each appointment must have a priority (a string),
a date and a description (a string).

\paragraph{PIMContact}

Contact items must be PIMEntites defined in a class named PIMContact. Each contact item must have a priority (a string), and strings for each of the following: first name, last name, email address.

There is one additional requirement on the implementation of the 4 item classes listed above, the 2 classes that involve a date must share an interface that you define. You must formally create this interface and have both PIMAppointment and PIMTodo implement this interface.

\paragraph{PIMManager}

You must also create a class named PIMManager that includes a main and provides some way of creating and managing items (from the terminal). You must support the following commands (functionality):

\begin{itemize}
\item \textbf{List}: \textit{print a list of all PIM items}
\item \textbf{Create}: \textit{add a new item}
\item \textbf{Save}: \textit{save the entire list of items}
\item \textbf{Load}: \textit{read a list of items from a file}
\end{itemize}

When creating a new item it is expected that the user must response to a sequence of prompts to enter the appropriate information
(and even to indicate what kind of item is being created). Do this any way you want, just make sure that your system provides enough
information (instructions) so that we can use your systems!

\subsection{Design}
\label{sec:pim:design}

Firstly, I need a class from which each kind of the item of the PIM will inherit, and that class is called
\lstinline|PIMEntity|.

For such a class, there are the method to get item type for reflect, the priority, and the owner.
\lstinline|PIMEntity| is an abstract class, although all methods are instanced.

There are also two important methods which are about serialization and deserialization with gson in JSON.

And for each kind of the item, the remainder things to add are just the information that the item will be with.

For the note, to-do list, and appointment the important thing is the context of the item where will hold the note, things to be done, or the appointments. The contact item will hold the information of your friends or somebody also.

The to-do list and appointment also inherit from the interface
\lstinline|IPIMDateable| and this interface is about whether a class has a headline.
 

Then the hold system need a class, named \lstinline|PIMCollection|,
is an instance of the \lstinline|Collection| where the items will be hold in. When we need to store the items to file, remote repo, database, or something also, we need an interface in which there are two method named \lstinline|push| and \lstinline|pull|. The method \lstinline|push| and \lstinline|pull| will be the operation to send to or fetch from the items with the place where the items stored.

Finally, we define a class to operate with IO and files, and a class to manage the system.

In the class \lstinline|PIMManage| where manage the system, we define a sub-class, named \lstinline|Opt|. This class define the opt
of the manager such as \verb|create| and \verb|save|.





\section{Substring}
\label{sec:sub}

\subsection{Task}
\label{sec:sub:task}

Create a class named Substring that will expect the first command line argument to be a string,
and the second two command line arguments to be integers, the first will be used as an index and the second as a length.
The output should be the subtring of string starting at the index and of the specified length.


\subsection{Design}
\label{sec:sub:design}

The design is simply simple. The things need to do include:

\begin{enumerate}
    \item Get the string from the command line parameters.
    \item Get the index of the substring's begin and end.
    \item Using the api of the \lstinline[language=Java]|String|
        to get the substring.
   \end{enumerate} 


\section{Calendar}
\label{sec:cal}

\subsection{Task}
\label{sec:cal:task}

Write a java program named cal (the main() should be in a class named "cal") that will print out to standard output the calendar for any month.
Your program should look at the command line arguments for a numeric month and year,
and print the calendar for that month in the format displayed below. If there are no command line arguments,
or if the command line arguments are not a valid month and year, your program must print the calendar for the current month.
This program is a java version of the Unix "cal" command.

\subsection{Design}

The things need to do include: getting the month wanted, making sure the first day's week, and printing.

Find out the first day's week, and then skip the empty positions. Then print the each of the day.




\end{document}

