\documentclass{article}

\usepackage{xeCJK}
\setCJKmainfont{SimSun}

\usepackage{listings}
\lstset{breaklines;numbers=left}

\title{Java Homework Report II}
\author{李约瀚 \\ 14130140331 \\ Qinka \\ qinka@live.com \\ me@qinka.pro \\ qinka@qinka.pw (remove soon)}


\begin{document}

\maketitle
\newpage

{\Huge \textbf{Copyleft (C) 2017}}
\par
{\Huge \textbf{GPL 3}}
\newpage

\tableofcontent
\newpage

\section{Persional Information System}
\label{sec:pim}

\subsection{Task}
\label{sec:pim:task}

This assignment involves the creation of simple Personal Information Management system that can deal with 4 kinds of items:
todo items, notes, appointments and contacts. Each of these kinds of items is described in more detail below.
The assignment requires that you create a class for each item type, and that each class extends an abstract base class provided for you.
In addition to creating the four classes, you need to create a manager class that supports some simple text-based commands
for creating and managing items.

\paragraph{PIMTodo}

Todo items must be PIMEntites defined in a class named PIMTodo. Each todo item must have a priority (a string),
a date and a string that contains the actual text of the todo item.

\paragraph{PIMNote}

Note items must be PIMEntites defined in a class named PIMNote. Each note item must have a priority (a string),
and a string that contains the actual text of the note.

\paragraph{PIMAppointment}

Appointment items must be PIMEntites defined in a class named PIMAppointment. Each appointment must have a priority (a string),
a date and a description (a string).

\paragraph{PIMContact}

Contact items must be PIMEntites defined in a class named PIMContact. Each contact item must have a priority (a string), and strings for each of the following: first name, last name, email address.

There is one additional requirement on the implementation of the 4 item classes listed above, the 2 classes that involve a date must share an interface that you define. You must formally create this interface and have both PIMAppointment and PIMTodo implement this interface.

\paragraph{PIMManager}

You must also create a class named PIMManager that includes a main and provides some way of creating and managing items (from the terminal). You must support the following commands (functionality):

\begin{itemize}
\item \textbf{List}: \textit{print a list of all PIM items}
\item \textbf{Create}: \textit{add a new item}
\item \textbf{Save}: \textit{save the entire list of items}
\item \textbf{Load}: \textit{read a list of items from a file}
\end{itemize}

When creating a new item it is expected that the user must response to a sequence of prompts to enter the appropriate information
(and even to indicate what kind of item is being created). Do this any way you want, just make sure that your system provides enough
information (instructions) so that we can use your systems!

\subsection{Design}
\label{sec:pim:design}






\section{Substring}
\label{sec:sub}

\subsection{Task}
\label{sec:sub:task}

Create a class named Substring that will expect the first command line argument to be a string,
and the second two command line arguments to be integers, the first will be used as an index and the second as a length.
The output should be the subtring of string starting at the index and of the specified length.







\section{Calendar}
\label{sec:cal}

\subsection{Task}
\label{sec:cal:task}

Write a java program named cal (the main() should be in a class named "cal") that will print out to standard output the calendar for any month.
Your program should look at the command line arguments for a numeric month and year,
and print the calendar for that month in the format displayed below. If there are no command line arguments,
or if the command line arguments are not a valid month and year, your program must print the calendar for the current month.
This program is a java version of the Unix "cal" command.








\end{document}

